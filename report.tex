%%%%%%%%%%%%%%%%%%%%%%%%%%%%%%%%%%%%%%%%%%%%%%%%%%%%%%%%%%%%%%%
%% OXFORD THESIS TEMPLATE

%%%%% CHOOSE PAGE LAYOUT

% This one will format for two-sided binding (ie left and right pages have mirror margins; blank pages inserted where needed):
%\documentclass[a4paper,twoside]{ociamthesis}
% This one will format for one-sided binding (ie left margin > right margin; no extra blank pages):
\documentclass[hidelinks, a4paper]{ociamthesis}
% This one will format for PDF output (ie equal margins, no extra blank pages):
%\documentclass[a4paper,nobind]{ociamthesis}



%%%%% SELECT YOUR DRAFT OPTIONS %TODO: turn off before submitting
% This adds a "DRAFT" footer to every normal page.  (The first page of each chapter is not a "normal" page.)
%\fancyfoot[C]{\emph{ENTWURF vom \today}}

% This highlights (in blue) corrections marked with (for words) \mccorrect{blah} or (for whole
% paragraphs) \begin{mccorrection} . . . \end{mccorrection}.  This can be useful for sending a PDF of
% your corrected thesis to your examiners for review.  Turn it off, and the blue disappears.
%\correctionstrue


%%%%% BIBLIOGRAPHY SETUP
% The science-type option: numerical in-text citation with references in order of appearance.
\usepackage[style=numeric-comp, sorting=none, backend=biber, doi=false, isbn=false]{biblatex}
% The humanities-type option: author-year in-text citation with an alphabetical works cited.
%\usepackage[style=authoryear, sorting=nyt, backend=biber, maxcitenames=2, useprefix, doi=false, isbn=false]{biblatex}
% custom
%\usepackage[style=authoryear, sorting=nyt, backend=biber, maxcitenames=2, useprefix, doi=true, isbn=true]{biblatex}

% This enables different categories, e.g. primary and secondary sources, papers read but not cited etc. --> don't forget to activate the translations within ociamthesis.cls
% Titles (headings) are adjusted within ociamthesis.cls; resources are assigned within configuration/references-mapping.tex
\newcommand*{\bibtitle}{References}
\DeclareBibliographyCategory{primary}
%\newcommand*{\bibtitlesecondary}{Secondary References}
%\DeclareBibliographyCategory{secondary}
%\newcommand*{\bibtitlereadbutnotcited}{Bibliography}
%\DeclareBibliographyCategory{readbutnotcited}

% This makes the bibliography left-aligned (not 'justified') and slightly smaller font.
\renewcommand*{\bibfont}{\raggedright\small}

% Change this to the name of your .bib file (usually exported from a citation manager like Zotero or EndNote).
\addbibresource{references.bib}


% Uncomment this if you want equation numbers per section (2.3.12), instead of per chapter (2.18):
%\numberwithin{equation}{subsection}



%%%%% YOUR OWN PERSONAL MACROS
% This is a good place to dump your own LaTeX macros as they come up.

% To make text superscripts shortcuts
	\renewcommand{\th}{\textsuperscript{th}} % ex: I won 4\th place
	\newcommand{\nd}{\textsuperscript{nd}}
	\renewcommand{\st}{\textsuperscript{st}}
	\newcommand{\rd}{\textsuperscript{rd}}

%%%%% THE ACTUAL DOCUMENT STARTS HERE
\begin{document}



%%%%% CHOOSE YOUR LINE SPACING HERE
% This is the official option.  Use it for your submission copy and library copy:
%\setlength{\textbaselineskip}{22pt plus2pt}
% This is closer spacing (about 1.5-spaced) that you might prefer for your personal copies:
\setlength{\textbaselineskip}{18pt plus2pt minus1pt}
%\setlength{\textbaselineskip}{18pt plus2pt}

% You can set the spacing here for the roman-numbered pages (acknowledgements, table of contents, etc.)
\setlength{\frontmatterbaselineskip}{17pt plus1pt minus1pt}

% Leave this line alone; it gets things started for the real document.
\setlength{\baselineskip}{\textbaselineskip}

%%%%% CHOOSE YOUR SECTION NUMBERING DEPTH HERE
% You have two choices.  First, how far down are sections numbered?  (Below that, they're named but
% don't get numbers.)  Second, what level of section appears in the table of contents?  These don't have
% to match: you can have numbered sections that don't show up in the ToC, or unnumbered sections that
% do.  Throughout, 0 = chapter; 1 = section; 2 = subsection; 3 = subsubsection, 4 = paragraph...

% The level that gets a number:
\setcounter{secnumdepth}{2}
% The level that shows up in the ToC:
\setcounter{tocdepth}{2}


%%%%% ABSTRACT SEPARATE
% In Oxford, a separate page with the abstract must be handed in to the Examination Schools. This is what this toggle is for.
%\newgeometry{margin=3.3cm}
\thispagestyle{empty}
\begin{alwayssingle}
    \begin{center}

    {\Large\bfseries Sample Title of the Thesis \par}
    {\large \vspace*{2ex} Your Name \par}
    {\large
        \vspace*{1ex}
        {Your College} \\
        {University of Oxford} \\
        \vspace*{1ex}
        {\it A thesis submitted for the degree of} \\
        {\it Degree Name} \\
        \vspace*{2ex}
        Month Year or Term \par
    }

        \vfill
        {\Large \bfseries Abstract}
    \end{center}

    \vspace{1ex}

    % Set a small baselineskip for the abstract content
    \setlength{\baselineskip}{0pt}

    % Your abstract content goes here
    \noindent
    \input{content/text/abstract}

    \vfill
\end{alwayssingle}
\restoregeometry

\setlength{\baselineskip}{\textbaselineskip}


\newcounter{roman}
% JEM: Pages are roman numbered from here, though page numbers are invisible until ToC.  This is in keeping with most typesetting conventions.
\begin{romanpages}

	%%%%% TITLE PAGE
	\include{functional-pages/title-thm-report}
	\clearpage
	\thispagestyle{empty}


	%%%%% PRE-TOC
	%\include{functional-pages/dedication}
	\begin{non-disclosure-notice}
    Der vorliegende Projekstudiumsbericht beinhaltet interne und vertrauliche Informationen der Musterfirma.
    Die Weitergabe und das Anfertigen von Abschriften des Inhalts dieser Arbeit ist im Gesamten als auch in Teilen untersagt.\\

    Ausnahmen bedürfen einer schriftlichen Genehmigung der Musterfirma.
\end{non-disclosure-notice}
%This essay contains internal and confidential information of the Viessmann Climate Solutions SE.
%Third parties are not permitted to access this work without the express permission
%of the named company. The information contained in this document may not be made publicly available unless expressly authorized by the company.
	%\begin{acknowledgements}
    \subsection*{Personal}

    This is where you thank your advisor, colleagues, and family and friends.


    \subsection*{Institutional}

    If you want to separate out your thanks for funding and institutional support, I don't think there's any rule against it.
    Of course, you could also just remove the subsections and do one big traditional acknowledgement section.
\end{acknowledgements}

%Danksagung
%Zusammenfassung:
%
%Ausdruck des Dankes an Personen und Institutionen.
%Erwähnung von Unterstützern und Förderern der Arbeit.
%Anerkennung von fachlicher und persönlicher Hilfe.
%Kein formeller Bestandteil der wissenschaftlichen Arbeit.
%Persönlich gehalten, ohne tiefer auf Inhalte einzugehen.
%Kein Datum und keine Unterschrift erforderlich.
%
%Englische Überschrift:
%
%Acknowledgements % Danksagung
	%\begin{preliminary-remarks}
    Bla.
\end{preliminary-remarks}

%Vorbemerkung
%Zusammenfassung:
%
%    Kurze Einführung mit persönlichen Kommentaren zur Arbeit.
%    Erläuterung von Einschränkungen und Schwierigkeiten während der Erstellung.
%    Angaben über methodische Entscheidungen und zeitliche Beschränkungen.
%    Hinweis auf spezielle Herausforderungen.
%    Keine Unterschrift, kein Datum erforderlich.
%
%Englische Überschrift:
%
%Preliminary Remarks % Vorbemerkung
	%\begin{preface}
    Bla.

    \vspace{4em}
    \noindent \includegraphics[width=70mm, height=12mm, keepaspectratio]{functional-pages/figures/simon-signature-black.pdf}\\
    \noindent Simon Vöhl\\
    Allendorf (Eder), den \today
\end{preface}

%Vorwort
%Zusammenfassung:
%
%    Persönlicher Kommentar zur Entstehung und Bedeutung der Arbeit.
%    Erklärung der Motivation und Ziele des Verfassers.
%    Danksagungen an Personen und Institutionen.
%    Formell mit Ort, Datum und Unterschrift versehen.
%    Hintergrundinformationen zur Themenwahl und Methodik.
%    Datum, Ort und Unterschrift erforderlich.
%
%Englische Überschrift:
%
%Preface % Vorwort
	\begin{abstract}
    \input{content/text/abstract}
\end{abstract}

	%%% PREPARATIONS
	% This lays the groundwork for per-chapter, mini tables of contents.  Comment the following line
	% (and remove \minitoc from the chapter files) if you don't want this.  Un-comment either of the
	% next two lines if you want a per-chapter list of figures or tables.
	%\dominitoc % include a mini table of contents
	%\dominilof  % include a mini list of figures
	%\dominilot  % include a mini list of tables

	% This aligns the bottom of the text of each page.  It generally makes things look better.
	\flushbottom

	%%%%% TOC AND OTHER LISTINGS
	\tableofcontents
	\newpage
	\listoffigures
	\mtcaddchapter %needed when adding a non-chapter (but chapter-like) entity to avoid confusing minitoc
	\newpage
	\listoftables
	\mtcaddchapter
	\newpage
	%\listof{code}{Codeverzeichnis}
	%\addcontentsline{toc}{chapter}{Codeverzeichnis}
	%	\mtcaddchapter


	%%%%% GLOSSARIES
	% Define the glossaries
\newglossaryentry{ai}{
    type=\acronymtype,
    name={AI},
    description={Artificial Intelligence},
    first={Artificial Intelligence (AI)},
    plural={AIs},
    firstplural={Artificial Intelligences (AIs)}
}

\newglossaryentry{ml}{
    type=\acronymtype,
    name={ML},
    description={Machine Learning},
    first={Machine Learning (ML)},
    plural={MLs},
    firstplural={Machine Learnings (MLs)}
}

\newglossaryentry{latex}{
    name=LaTeX,
    description={A document preparation system}
}

\newglossaryentry{bibtex}{
    name=BibTeX,
    description={A tool for formatting lists of references}
}

\newglossaryentry{alpha}{
    type=symbols,
    name=\ensuremath{\alpha},
    description={Alpha symbol used in mathematics and science}
}

\newglossaryentry{beta}{
    type=symbols,
    name=\ensuremath{\beta},
    description={Beta symbol used in mathematics and science}
}
	\newpage
	\printunsrtglossary[type=main, title={Glossar}] % Glossary
	\mtcaddchapter % Avoid confusion with minitoc

	\printunsrtglossary[type=\acronymtype, title={Abkürzungsverzeichnis}] % List of Abbreviations
	\mtcaddchapter % Avoid confusion with minitoc

	\printunsrtglossary[type=symbols, title={Symbolverzeichnis}] % List of Symbols
	\mtcaddchapter % Avoid confusion with minitoc

	\glsaddall % Ensure entries are shown right away % TODO remove for final document

	% The Roman pages, like the Roman Empire, must come to its inevitable close.
	\setcounter{roman}{\value{page}}
\end{romanpages}

%%%%% CONTENT
\flushbottom
\chapter{Template}\label{ch:template}


\section{Section A}%\label{sec:a}
This is regular text.

\subsection{Section A-1}
This is regular text.\cite{DINBEISPIEL}

\subsection{Section A-2}
\enquote{And I am a cited text!}\parencite[vgl.][S. 152]{sousa_new_2005}

\subsubsection{Section A-2-i}
\verb|some_file.ending|
This is an \mccorrect{inline} correction (highlighting can be disabled in report.tex).

\begin{mccorrection}
    This is a whole paragraph which is being highlighted.
\end{mccorrection}

Just some \textit{italic} text.
Just some \textbf{bold} text.
Just some \texttt{very bold} text.

\section{Section B}%\label{sec:b}
This is regular text.

\begin{figure}
    \centering
    \includegraphics[width=0.7\textwidth]{content/figures/sample/Gray498}
    \caption[I'm the short title for the TOC]{And I'm the caption appearing next to the actucal graphic. So I can be quite extensive, and, even if not very sensible, completely distinct from the TOC variant.}
    \label{fig:heart}
\end{figure}

\begin{code}[H]
    \begin{minted}{py3}
if _algorithm == "MD5" or _algorithm == "MD5-SESS":
    def md5_utf8(x):
        if isinstance(x, str):
            x = x.encode("utf-8")
        return hashlib.md5(x).hexdigest()
    hash_utf8 = md5_utf8
    \end{minted}
    \caption[Verwendung Algorithmus]{Verwendung eines Algorithmus mit extrem langem und nicht ins TOC passendem Titel}
    \label{code:md5}
\end{code}
\input{content/text/ch1-intro}
\input{content/text/ch2-litreview}

%%%%% POST-CONTENT
\begin{romanpages}
	\setcounter{page}{\value{roman}+1}
	%%%%% MAPPING
% SOURCES <> REFERENCE CATEGORY

\addtocategory{primary}{harvey_exercitatio_1628}
%\addtocategory{secondary}{harvey_exercitatio_1628}
%\addtocategory{readbutnotcited}{harvey_exercitatio_1628}
	%%%%% ATTENTION: These comments refer to how I would use the categories primary/secondary/...

%%%%% REFERENCES /// DIRECTLY CITED


\setlength{\baselineskip}{0pt} % JEM: Single-space References

{\renewcommand*\MakeUppercase[1]{#1}%
\printbibliography[heading=bibintoc,title={\bibtitle}]} \mtcaddchapter
%\printbibliography[heading=bibintoc,title={\bibtitle}, category=primary]} \mtcaddchapter



%%%%% SECONDARY REFERENCES /// INDIRECT CITATION



%\setlength{\baselineskip}{0pt} % JEM: Single-space References

%{\renewcommand*\MakeUppercase[1]{#1}%
%\printbibliography[heading=bibintoc,title={\bibtitlesecondary},category=secondary]} \mtcaddchapter



%%%%% BIBLIOGRAPHY /// READ BUT NOT CITED LITERATURE


%\setlength{\baselineskip}{0pt} % JEM: Single-space References

%{\renewcommand*\MakeUppercase[1]{#1}%
%\printbibliography[heading=bibintoc,title={\bibtitlereadbutnotcited},category=readbutnotcited]} \mtcaddchapter
	\begin{declaration-of-authorship}
    Ich versichere, dass ich diese Arbeit selbstständig verfasst und keine anderen als
    die angegebenen Hilfsmittel benutzt habe. Die den benutzten Hilfsmitteln wörtlich
    oder inhaltlich entnommenen Stellen habe ich unter Quellenangaben kenntlich gemacht. Ich bin in vollem Umfang für die Eigenständigkeit, den Inhalt und die Qualität
    meiner Arbeit verantwortlich. Ich bleibe auch bei Verwendung von Hilfsmitteln, die
    Künstliche Intelligenz (KI) beinhalten können, in vollem Umfang verantwortlich für
    meine Argumentation, Aussagen, Übersetzungen, Zitate und Quellenangaben, insbesondere für deren Richtigkeit, Sachangemessenheit, wissenschaftliche Anerkennung, Originalität und Aktualität. Die Verwendung solcher Hilfsmittel wurde vorab
    mit dem Partnerunternehmen abgestimmt. Die im Literaturverzeichnis angegebenen Quellen habe ich vor dem Hintergrund des zitierten Sachverhalts im Original
    gelesen und gemäß den Grundsätzen wissenschaftlichen Arbeitens überprüft.
    
    Die Arbeit hat in gleicher oder ähnlicher Form noch keiner anderen Prüfungsbehörde vorgelegen.

    \vspace{4em}
    \noindent Max Mustermann\\
    Musterstadt, den \today
\end{declaration-of-authorship}

	%%%%% APPENDICES %%
	% Starts lettered appendices, adds a heading in table of contents, and adds a
	%    page that just says "Appendices" to signal the end of your main text.
	%\startappendices
	%\setlength{\baselineskip}{\textbaselineskip} % necessary (idk, somewhere (prob. references) the baselineskip is altered...)
	%\include{content/text/appendix-1}


	%%%%% REFERENCES

\end{romanpages}
\end{document}
